\chapter{Fourier transform of Green's function}

\section{Fourier transform of the regularized Green's function}

\begin{definition}
  \label{def:Green_Fourier_transf}
  \uses{def:Green_function}
  Let $0 \le r < 1$.
  The \textbf{Fourier transform} of the regularized Green's function
  $\Greg{r} \, \colon \, \bZ^d \to \R$
  is the function
  \begin{align*}
  \Freg{r} \, \colon \, \bR^d \to \C
  \end{align*}
  given by
  \begin{align*}
  \Freg{r} (\theta) = \sum_{x \in \bZ^d} e^{\ii x \cdot \theta} \, \Greg{r}(x) .
  \end{align*}
\end{definition}

\section{Explicit formula for the Fourier transform}

\section{Inversion of the discrete Fourier transform}

\begin{lemma}
  \label{lem:Green_Fourier_inverse}
  \uses{def:Green_Fourier_transf}
  For any $x \in \bZ^d$ and $0 \le r < 1$, we have
  \begin{align*}
  \Greg{r} (\theta)
  = \frac{1}{(2\pi)^d} \iint_{\Fbox} e^{-\ii x \cdot \theta} \, \Freg{r} (\theta) \, \ud^d \theta .
  \end{align*}
\end{lemma}
\begin{proof}
\ldots
\end{proof}

Recall that we are interested in $\EX[L]$, where $L$ is the number of visits to the
origin by the random walk. Lemma~\ref{lem:Green_function_nonregularized_limit}
states that $\EX[L]$ is the increasing limit of $\Greg{r}(\vec{0})$ as $r \nearrow 1$,
and Lemma~\ref{lem:Green_Fourier_inverse} gives a formula for $\Greg{r}(\vec{0})$
in terms of the Fourier transform.

% This allows to express the
% \begin{corollary}
%   \label{lem:Green_Fourier_inverse}
%   \uses{def:Green_Fourier_transf}
%   For any $x \in \bZ^d$ and $0 \le r < 1$, we have
%   \begin{align*}
%   \Greg{x} (\theta)
%   = \frac{1}{(2\pi)^d} \iint_{\Fbox} e^{-\ii x \cdot \theta} \, \Freg{r} (\theta) \, \ud^d \theta .
%   \end{align*}
% \end{corollary}
% \begin{proof}
% \ldots
% \end{proof}
