\chapter{P\'olya's theorem}

The goal of this project is to prove that:
\begin{quote}
A drunk man will find his way home, but a drunk bird may get lost forever.

\emph{--- Shizuo Kakutani}

\url{https://mathshistory.st-andrews.ac.uk/Biographies/Kakutani/quotations/}
\end{quote}


Somewhat more mathematically, the goal is the following:

\begin{theorem}[P\'olya's theorem]
  \label{thm:Polya}
  %\uses{def:random_walk}
  %\lean{RW.measurable}
  The simple random walk $\RW = \big( \RW(t) \big)_{t \in \bN}$
  on the $d$-dimensional grid $\bZ^d$ is recurrent if $d \le 2$
  and transient if $d \, > \, 2$.
%   Let $\RW = \big( \RW(t) \big)_{t \in \bN}$ be a simple random walk
%   on the $d$-dimensional grid $\bZ^d$.
\end{theorem}
\begin{proof}
\uses{lem:recurrent_iff_expectation_recurrent, thm:Polya_alt}
Theorem~\ref{thm:Polya_alt} below asserts that $X$
is expectation recurrent iff $d \le 2$,
and Lemma~\ref{lem:recurrent_iff_expectation_recurrent}
shows that recurrence and expectation recurrence are equivalent
for the simple random walk.
\end{proof}

The essence of the proof is to establishe the following slightly modified
version of the theorem.

\begin{theorem}[P\'olya's theorem, alternative form]
  \label{thm:Polya_alt}
  %\uses{def:random_walk}
  %\lean{RW.measurable}
  The simple random walk $\RW = \big( \RW(t) \big)_{t \in \bN}$
  on the $d$-dimensional grid $\bZ^d$ is expectation recurrent if $d \le 2$
  and expectation transient if $d \, > \, 2$.
%   Let $\RW = \big( \RW(t) \big)_{t \in \bN}$ be a simple random walk
%   on the $d$-dimensional grid $\bZ^d$.
\end{theorem}
\begin{proof}
\uses{lem:Green_function_nonregularized_limit}
\ldots
\end{proof}
