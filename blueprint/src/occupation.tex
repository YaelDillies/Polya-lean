\chapter{Occupations and Green's functions of random walks}



\section{Regularized occupation}


\subsection*{Regularized occupation of a walk}

\begin{definition}
  \label{def:walkRegOccup}
  \lean{walkRegularizedOccupation}
  \uses{def:walk}
  \leanok
  Let $\walk \colon \bN \to \bZ^d$ be a walk and let $r \ge 0$.
  Then the \textbf{$r$-regularized occupation}
  of $\walk$ at $x \in \bZ^d$ is
  \begin{align}\label{eq: reg walk occupation}
  L^{\walk}_r(x) = \sum_{t \in \bN} r^t \, \indOf{\set{\walk(t) = x}} .
  \end{align}
\end{definition}

\begin{lemma}
  \label{lem:walkRegOccup_incr}
  \uses{def:walkRegOccup}
  \lean{walkRegularizedOccupation_mono}
  The $r$-regularized occupation $L^{\walk}_r(x)$
  of a walk $\walk$ is an increasing
  function of $r$.
\end{lemma}
\begin{proof}
Each term in the defining sum \eqref{eq: reg walk occupation}
is increasing.
\end{proof}

\begin{lemma}
  \label{lem:walk_regOcc_geometric_bound}
  \uses{def:walkRegOccup}
  \lean{walkRegularizedOccupation_le}
  The $r$-regularized occupation $L^{\walk}_r(x)$
  of a walk $\walk$ at any point $x \in \bZ^d$
  satisfies $L^{\walk}_r(x) \le \frac{1}{1-r}$.
\end{lemma}
\begin{proof}
Calculate, using the definition \eqref{eq: reg walk occupation}:
\begin{align*}
L^{\walk}_r(x) = \sum_{t \in \bN} r^t \, \indOf{\set{\walk(t) = x}}
    \le \sum_{t \in \bN} r^t = \frac{1}{1-r}.
\end{align*}
\end{proof}

\begin{lemma}
  \label{lem:sum_walk_regOcc_eq_geom_ser}
  \uses{def:walkRegOccup}
  \lean{tsum_walkRegularizedOccupation_eq_geom_series}
  The sum over all points $x \in \bZ^d$ of the $r$-regularized
  occupations $L^\walk_r(x)$ of a walk $\walk \colon \bN \to \bZ^d$ is
  \begin{align*}
  \sum_{x \in \bZ^d} L^\walk_r(x) = \sum_{t \in \bN} r^t .
  \end{align*}
  (Both sides above are infinite if $r \ge 1$, but the equality
  nevertheless holds in $[0,+\infty]$.)
\end{lemma}
\begin{proof}
Unravel the definitions and interchange the two summations by Fubini's theorem:
\begin{align*}
\sum_{x \in \bZ^d} L^\walk_r(x)
= & \; \sum_{x \in \bZ^d} \sum_{t \in \bN} r^t \, \indOf{\set{\walk(t) = x}} \\
= & \; \sum_{t \in \bN} \underbrace{\sum_{x \in \bZ^d} r^t \, \indOf{\set{\walk(t) = x}}}_{=r^t}.
\end{align*}
Here we noticed in the last expression that since at time $t$ the walk is
at exactly one point, there is exactly one value of~$x$ which contributes
to the sum over~$x$, and this contribution is $r^t$.
\end{proof}


\subsection*{Regularized occupation of a random walk}

\begin{definition}
  \label{def:regularized occupation}
  \lean{regularizedOccupation}
  \uses{def:walkRegOccup}
  \leanok
  Let $\RW$ be a random walk on $\bZ^d$ and let $r \ge 0$.
  Then the \textbf{$r$-regularized occupation}
  of $\RW$ at $x \in \bZ^d$ is
  \begin{align}\label{eq: reg occupation}
  L_r(x) = L^{\RW}_r(x) = \sum_{t \in \bN} r^t \, \indOf{\set{\RW(t) = x}} .
  \end{align}
\end{definition}

\begin{lemma}
  \label{lem:regularized occupation_mble}
  \uses{def:regularized occupation}
  \lean{regularizedOccupation.measurable}
  The regularized occupation $L_r(x)$ of a random walk $RW$
  is a $[0,+\infty]$-valued random variable.
\end{lemma}
\begin{proof}
\uses{lem:RW_mble}
\ldots
\end{proof}

\begin{lemma}
  \label{lem:regularized_occupation_incr}
  \uses{def:regularized occupation}
  \lean{regularizedOccupation_mono}
  The $r$-regularized occupation $L_r(x)$
  of a random walk $\RW$ is increasing in $r$.
\end{lemma}
\begin{proof}
\uses{lem:walkRegOccup_incr}
This follows from Lemma~\ref{lem:walkRegOccup_incr}.
\end{proof}

\begin{lemma}
  \label{lem:regularized_occupation_geometric_bound}
  \uses{def:regularized occupation}
  \lean{regularizedOccupation_le}
  The $r$-regularized occupation $L_r(x)$
  of a random walk $\RW$ at any point $x \in \bZ^d$
  satisfies $L_r(x) \le \frac{1}{1-r}$.
\end{lemma}
\begin{proof}
\uses{lem:walk_regOcc_geometric_bound}
Follows from Lemma~\ref{lem:walk_regOcc_geometric_bound}.
\end{proof}

\begin{lemma}
  \label{lem:regularized_occupation_left_cont}
  \uses{def:regularized occupation}
  \lean{regularizedOccupation_apply_tendsto_of_monotone}
  If $(r_n)_{n \in \bN}$ is an increasing sequence with
  limit $r = \lim_{n \to \infty} r_n$, then
  for any $x \in \bZ^d$ the random variables $L_{r_n}(x)$
  have limit $L_{r}(x) = \lim_{n \to \infty} L_{r_n}(x)$.
\end{lemma}
\begin{proof}
\uses{lem:walkRegOccup_incr}
This uses a monotone convergence theorem:
each term $r_n^t \, \indOf{\set{\RW(t) = x}}$ of the infinite
sum~\eqref{eq: reg occupation} is increasing and has
limit $r^t \, \indOf{\set{\RW(t) = x}}$ (by continuity of the
$t$-power function $\rho \mapsto \rho^t$), so the sum is
increasing and the limit of the sum is the sum of the limits.
\end{proof}

\begin{lemma}
  \label{lem:sum_regularized_occupation_eq_geom_ser}
  \uses{def:regularized occupation}
  \lean{tsum_regularizedOccupation_eq_geom_series}
  The sum over all points $x \in \bZ^d$ of the $r$-regularized
  occupations $L_r(x)$ of a random walk $\RW$ is
  \begin{align*}
  \sum_{x \in \bZ^d} L_r(x) = \sum_{t \in \bN} r^t .
  \end{align*}
  (Both sides above are infinite if $r \ge 1$, but the equality
  nevertheless holds in $[0,+\infty]$.)
\end{lemma}
\begin{proof}
\uses{lem:sum_walk_regOcc_eq_geom_ser}
This is a random walk version of Lemma~\ref{lem:sum_walk_regOcc_eq_geom_ser},
but the right hand side expression does not depend on the walk, so there
is no randomness involved (the right hand side is a constant random variable).
\end{proof}

\begin{lemma}
  \label{lem:sum_regularized_occupation_eq}
  \uses{def:regularized occupation}
  \lean{tsum_toReal_regularizedOccupation_eq}
  If $r<1$, then the infinite sum in
  Lemma~\ref{lem:sum_regularized_occupation_eq_geom_ser} is convergent
  in~$\bR$, and the equality
  \begin{align*}
  \sum_{x \in \bZ^d} L_r(x) = \frac{1}{1-r}
  \end{align*}
  holds in $\bR$.
\end{lemma}
\begin{proof}
\uses{lem:sum_regularized_occupation_eq_geom_ser}
Apply Lemma~\ref{lem:sum_regularized_occupation_eq_geom_ser}
and the sum of a geometric series.
\end{proof}

\begin{lemma}
  \label{lem:sum_expected_regularized_occupation_bound}
  \uses{def:regularized occupation}
  \lean{tsum_lintegral_norm_regularizedOccupation_le}
  If $r<1$, then the
  sum over points of the expected absolute values $|L_r(x)|$ of the regularized
  occupation of a random walk $\RW$ has the upper bound
  \begin{align*}
  \sum_{x \in \bZ^d} \EX \Big[ \big| L_r(x) \big| \Big] \le \frac{1}{1-r} 
  \end{align*}
\end{lemma}
\begin{proof}
\uses{lem:sum_regularized_occupation_eq}
Interchange the expected value and the sum by Fubini's theorem:
\begin{align*}
\sum_{x \in \bZ^d} \EX \Big[ \big| L_r(x) \big| \Big]
= \sum_{x \in \bZ^d} \EX \Big[ \big| L_r(x) \big| \Big] .
\end{align*}
Then notice that $|L_r(x)| = L_r(x)$, and apply
Lemma~\ref{lem:sum_regularized_occupation_eq}.
\end{proof}



\section{Green's function}

\begin{definition}
  \label{def:Green_function}
  \lean{regularizedG}
  \uses{def:regularized occupation}
  \leanok
  Let $\RW$ be a random walk on $\bZ^d$ and let $0 \le r < 1$.
  Then the \textbf{$r$-regularized Green's function}
  of $\RW$ is the function $\Greg{r} \colon \bZ^d \to \bR$
  whose value at $x \in \bZ^d$ is the expected value of the
  regularized occupation $L_r(x)$ at $x$,
  \begin{align}\label{eq: reg Green}
  \Greg{r}(x) = \EX \big[ L_r(x) \big] .
  \end{align}
\end{definition}

\begin{lemma}
  \label{lem:sum_Green_function_aux}
  \uses{def:Green_function}
  \lean{tsum_regularizedG_eq_lintegral_tsum}
  We have
  \begin{align*}
  \sum_{x \in \bZ^d} \Greg{r}(x) = \EX \Big[ \sum_{x \in \bZ^d} L_r(x) \Big] .
  \end{align*}
\end{lemma}
\begin{proof}
\uses{lem:sum_expected_regularized_occupation_bound}
Interchange the expected value and the sum by Fubini's theorem.
The integrability hypotheses for the appropriate version of Fubini's theorem
follow from Lemma~\ref{lem:sum_expected_regularized_occupation_bound}.
\end{proof}

\begin{lemma}
  \label{lem:sum_Green_function}
  \uses{def:Green_function}
  \lean{tsum_regularizedG_eq}
  We have
  \begin{align*}
  \sum_{x \in \bZ^d} \Greg{r}(x) = \frac{1}{1-r} .
  \end{align*}
\end{lemma}
\begin{proof}
\uses{lem:sum_Green_function_aux, lem:sum_regularized_occupation_eq}
First apply Lemma~\ref{lem:sum_Green_function_aux} to interchange the expected value
and the sum over~$x$. Then apply Lemma~\ref{lem:sum_regularized_occupation_eq}.
This yields
\begin{align*}
\sum_{x \in \bZ^d} \Greg{r}(x) = \EX \Big[ \sum_{x \in \bZ^d} L_r(x) \Big]
  = \EX \Big[ \frac{1}{1-r} \Big] = \frac{1}{1-r} ,
\end{align*}
where the last equality is just taking the expected value of a constant.
\end{proof}



\section{Expected occupation from the Green's function}

\begin{lemma}
  \label{lem:Green_function_nonregularized_limit}
  \uses{def:Green_function}
  Let $\RW = \big(\RW(t)\big)_{t \in \bN}$ be a random walk
  starting from the origin $\vec{0} \in \bZ^d$. Denote by
  $L = \# \set{t \in \bN \, \Big| \, X(t) = \vec{0}}$ the number of times
  the random walk is at the origin. Then we have
  \begin{align*}
  \Greg{r}(\vec{0}) \nearrow \EX [L] \qquad \text{ as } r \nearrow 1 .
  \end{align*}
\end{lemma}
\begin{proof}
\uses{lem:regularized_occupation_left_cont}
Recall that $\Greg{r}(\vec{0}) = \EX \big[ L_r(\vec{0}) \big]$ by definition,
and observe that $L = L_1(\vec{0})$. Therefore the statement is equivalent
to $\Greg{r}(\vec{0}) \nearrow \Greg{1} (\vec{0})$ as $r \nearrow 1$.
For this, it suffices to prove that
whenever $(r_n)_{n \in \bN}$ is an increasing sequence with
limit $1 = \lim_{n \to \infty} r_n$, then
$\Greg{r_n}(\vec{0}) \nearrow \Greg{1}(\vec{0})$ as $n \to \infty$.
Lemma \ref{lem:regularized_occupation_left_cont}
shows exactly that.
\end{proof}
